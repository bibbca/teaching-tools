\documentclass[11pt]{article}
\pagestyle{plain}
\usepackage[margin=1.0in,tmargin=0.75in,bmargin=0.75in]{geometry}
\usepackage{fancyhdr}
\pagestyle{fancy}
\fancyhf{}
\renewcommand{\headrulewidth}{0pt}
\renewcommand{\footrulewidth}{0pt}
\rfoot{\thepage}

\usepackage{amsmath,amssymb,graphicx,amsthm}
\providecommand{\abs}[1]{\left\lvert#1\right\rvert}
\providecommand{\norm}[1]{\left\lVert#1\right\rVert}
\usepackage{pgfplots}
\pgfplotsset{compat=1.13}

\newcommand{\tf}{[T/F] }
\newcommand{\bfN}{\mathbf{N}}

\usepackage[T1]{fontenc}
\usepackage{mathpazo} 
\setlength{\parindent}{0em}
\setlength{\parskip}{\baselineskip}

\DeclareMathOperator{\Span}{span}

\begin{document}
\title{Basic Probability Lab}
\section*{Algebra 2: Basic Probability Lab \hfill 9.2}
Name: \underline{\hspace{3in}} \hfill Date: \underline{\hspace{1in}}
\par
Our goal is to discover math concepts in a safe, real world environment.
\bigskip
\hrule
\section*{Coin Flipping}
\subsection*{Information}
Many people use the flip of a coin to make a wide variety of differing decisions and choices.  We will discover why they are convenient.\\
We call the result of an experiment or situation an \textbf{outcome}.

\subsection*{Critical Thinking Questions}
\begin{enumerate}
\item What possible outcomes are you able to get from a coin flip?
\item Is every outcome equally likely?
\item How can you tell if a coin is fair?
\end{enumerate}

\subsection*{Experiment}
\begin{enumerate}
\item Retrieve 10 coins from the teacher.
\item Flip these 10 coins, 10 times each for a total of 100 attempts.
\item Record your results as you go in the space below:
\vspace{1in}
\item Report your findings on the whiteboard.
\end{enumerate}

\pagebreak

\section*{Dice Rolling}
\subsection*{Information}
Dice are used in many popular and unpopular games such as Yahtzee, and Settlers of Catan (I'll let you decide which ones are which).  Dice come in all shapes and sizes.  Some popular ones include 12-sided die called dodecahedron and 20-sided die called Icosahedron.  They are named after the number of sides they have. 

\subsection*{Critical Thinking Questions}
\begin{enumerate}
\item What possible outcomes are you able to get from a dice roll?
\item Is every outcome equally likely?
\item If you were to roll a die 100 times and average all of your outcomes together, what would you expect to get?
\end{enumerate}

\subsection*{Experiment}
\begin{enumerate}
\item Retrieve a die from the teacher.
\item Roll this die 100 times.
\item Record your results as you go in the space below:
\vspace{2in}
\item Calculate the average of your outcomes in the space below: 
\vspace{2in}
\item Compare with a neighbor, is your average outcome the: same, greater, less?
\item Retrieve another die from the teacher.
\item Roll your 2 dice 100 times.
\item Record the sum of your dice as you go in the space below:
\vspace{2in}
\item Calculate the average of your outcomes in the space below: 
\vspace{2in}
\item Compare with a neighbor, is your average outcome the: same, greater, less?
\end{enumerate}
\vspace{1.5in}
How did you like this activity?
\par
0 \hfill 1 \hfill 2 \hfill 3 \hfill 4 \hfill 5\\
Not at all \hfill Best thing since sliced bread
\par
How can this activity be improved?
\end{document}