\documentclass[11pt]{article}
\pagestyle{plain}
\usepackage[margin=1.0in,tmargin=0.75in,bmargin=0.75in]{geometry}
\usepackage{fancyhdr}
\pagestyle{fancy}
\fancyhf{}
\renewcommand{\headrulewidth}{0pt}
\renewcommand{\footrulewidth}{0pt}
\rfoot{\thepage}

\usepackage{amsmath,amssymb,graphicx,amsthm}
\providecommand{\abs}[1]{\left\lvert#1\right\rvert}
\providecommand{\norm}[1]{\left\lVert#1\right\rVert}
\usepackage[T1]{fontenc}
\usepackage{mathpazo} 
\setlength{\parindent}{0em}
\setlength{\parskip}{\baselineskip}
\usepackage[table,xcdraw]{xcolor}
\begin{document}
\title{Probability Project}
\section*{Algebra 2: Probability Final Project}

Our goal is to discover math concepts in a safe, real world environment.
\bigskip
\hrule
\section*{Probability Final Project: Design Your Own Game}
\subsection*{Information}
In this assignment, you will be designing your own game.  This game should be one that you would play at a carnival, amusement park, or casino.  This game must be original (it cannot be a game that already exists).  You must be able to explain the probability of your game. (So don't make it too complicated.)
\par
Due Date: \textbf{May 24, 2018}
\subsection*{Final Products:}
\begin{enumerate}
\item Game - Include all game boards, playing pieces, cards, balls, etc. for your game.
\item Instructions - You must create a set of typed instructions to clearly explain the rules to your game and the win conditions.
\item Play Tests - At least 5 people play your game with at least 30 total wins or losses.
\item Write-Up - You will also include a typed response to the questions below. (2-6 pages recommended)
\end{enumerate}
Grading: See attached rubric.

\subsection*{The Write-Up}
This inclusion is to help guide your writeup.  The following questions are meant to help guide your writing.
\begin{enumerate}
\item Game Overview
\begin{itemize}
\item What type of game is it?
\item Where would you play this type of game?
\item How much would you charge to play it?
\item What are the prizes if you win?
\end{itemize}
\item Probability Analysis
\begin{itemize}
\item Is the game fair?  How do you know?
\item If the game is not fair, how could you change the game to make it fair?
\item What is a participant's expected value if they were to play the game?
\end{itemize}
\item Statistical Analysis
\begin{itemize}
\item Do your results show that your game is fair?
\item How sure are you that your game is fair?
\item Did people enjoy your game?
\item How did you choose who would play your game?
\item Would your game make money?  If so, how much?
\end{itemize}
\item Reflection
\begin{itemize}
\item What were your overall feelings about this project?
\item Did this project help you understand probability any better?
\item What have your learned about ``Fair Games''?
\item Would you advise people to play the Washington State lottery?
\end{itemize}
\end{enumerate}

\begin{table}[]
\small
\centering
%\caption{My caption}
\begin{tabular}{|p{1.5cm}|p{4cm}|p{4cm}|p{4cm}|p{2.5cm}|}
\hline
\rowcolor[HTML]{C0C0C0} 
Category                  & 4                                                                                                                                                                                                                           & 3                                                                                                                                                                                                                                & 2                                                                                                                                                                             & 1                                                                                                      \\ \hline
Game                      & The student creates a fully functioning game that people can play.  They bring all of the game materials to the submission date.                                                                                          & The students creates a game that people can play.  There may be slight oversights, but overall the game can be played.                                                                                                           & The student creates a game, but it cannot be played in class.                                                                                                                 & The student has an idea for a game.                                                                    \\ \hline
Instructions              & Instructions are clear and easy to follow.  The game can be played by others without referring to the game creator for help.                                                                                                & Instructions are somewhat clear and easy to follow.  The game can be played by others with minimal interaction with creators of the game.                                                                                         & Student has written instructions but they are unclear and a verbal description of the game is necessary.                                                                      & Instructions are incomplete.                                                                           \\ \hline
Writeup                   & Student has comprehensive write-up including: introduction, instructions, game description, probability analysis, statistical analysis, and reflection.  The write-up has been thoughtfully prepared and is well reflected. & Student has writeup including: introduction, instructions, game description, probability analysis, statistical analysis, and reflection.                                                                                         & Student has an incomplete write-up including some of the following: introduction, instructions, game description, probability analysis, statistical analysis, and reflection. & An attempt at a write-up has been made                                                                 \\ \hline
Probability Analysis      & The student provides accurate analysis of the math behind their game.  The idea of a fair game is clearly explained in terms of their project and an alternative for making their game fair is presented.                   & The student provides somewhat accurate analysis of the math behind their game.  The idea of a fair game is explained in terms of their project.  An attempt at providing an alternative for making their game fair is presented. & Student provides some analysis of the math behind their game.  The idea of a fair game is mentioned.                                                                          & Student attempts some sort of analysis of their probability.                                           \\ \hline
Statistical Analysis      & The student provides accurate analysis of the math interpretation of their gathered data.  Sampling techniques are exhibited.                                                                                               & The student provides somewhat accurate analysis of the math interpretation of their gathered data.  Sampling techniques are considered and discussed.                                                                            & Student provides some analysis of the gathered data.  Sampling techniques are mentioned.                                                                                      & Student attempts some sort of analysis of their gathered data.  Sampling techniques are not mentioned. \\ \hline
Reflection                & Reflection clearly explains the student's thought process during the project. The relevance of the project is clearly described.                                                                                            & Reflection attempts to explain the student's through process during the project.  The relevance of the project is described.                                                                                                     & Reflection attempts to explain the student's thought process during the project.                                                                                              & Some attempt at a reflection has been made.                                                            \\ \hline
Neatness and Organization & The work is presented in a neat, clear, organized fashion that is easy to read.                                                                                                                                             & The work is presented in a neat organized fashion that is usually easy to read.                                                                                                                                                  & The work is presented in an organized fashion but may be hard to read at times.                                                                                               & The work appears sloppy, rushed, or unorganized.  It is had to know what information goes together.    \\ \hline
\end{tabular}
\end{table}
\end{document}